\documentclass[11pt]{article} 
\usepackage{geometry} 
\geometry{letterpaper} 
\usepackage{graphicx} 
\usepackage{amssymb} 
\usepackage{amsmath} 
\usepackage{epstopdf} 
\usepackage{hyperref} 
\usepackage{natbib} 
\DeclareGraphicsRule{.tif}{png}{.png}{`convert #1 `dirname #1`/`basename #1 .tif`.png} 
\graphicspath{
{/Users/Andy/Cruises_Research/OceanMixingGroup/cruises/ctd_chipod/FLEAT/Figures/}
{/Users/Andy/Cruises_Research/ChiPod/FLEAT/Data/proc/Chipod/SNSN2014/figures/}
{/Users/Andy/Cruises_Research/ChiPod/FLEAT/Data/proc/Chipod/SNSN2002/figures/}
{/Users/Andy/Cruises_Research/ChiPod/FLEAT/Data/proc/Chipod/SNSN1013/figures/}
} 

\title{FLEAT CTD-Chipod Notes} 
\author{Andy Pickering} 
\begin{document} 
\maketitle 
\tableofcontents 
\newpage 

\section{About} 

Notes on processing and analysis of CTD-chipod data collected during FLEAT cruise.. 

\begin{figure}[htbp] 
\includegraphics[width=38pc]{FLEAT_kml_map.jpg} 
\caption{Map of CTD cast locations during cruise FLEAT.} 
\label{map} 
\end{figure} 

\section{Data and Processing} 

Data paths are set w/ :  
\begin{itemize} 
\item erb+Load_chipod_paths_FLEAT.m+ 
\end{itemize} 

Chipod deployment info is given in  
\begin{itemize} 
\item erb+Chipod_Deploy_Info_FLEAT+ 
\end{itemize} 

\subsection{CTD} 

Raw (hex) CTD data are processed with: 
\begin{itemize} 
\item erb+Process_CTD_hex_FLEAT.m+ 
\end{itemize} 

\subsection{Chipod} 

\begin{table}[htdp] 
\caption{Summary of Chipod deployment .} 
\begin{center} 
\begin{tabular}{|c|c|c|c|} 
\hline 
\end{tabular} 
\end{center} 
\label{default} 
\end{table}Raw chipod files are plotted with  
\begin{itemize} 
\item erb+PlotChipodDataRaw_FLEAT.m+ 
\end{itemize} 
to check for obvious issues/malfunctions with any of the instruments; these plots are saved in erb+/Figures/chipodraw/+.  
Based on a quick look at these: 
\begin{itemize} 
\item 
\end{itemize} 

Files with obviously bad or missing data (based on above raw plots) are noted in erb+bad_file_list_XXXX.m+ and, which will prevent them from being loaded in the processing. 

Processing scripts are: 
\begin{itemize} 
\item erb+MakeCasts_CTDchipod_FLEAT.m+ 
\item erb+DoChiCalc_FLEAT.m+ 
\end{itemize} 

\subsection{Further processing and analysis} 

\begin{itemize} 
\item Data from all casts are combined into a single structure w/ erb+MakeCombinedStruct_FLEAT.m+ 
\item  
\end{itemize} 

\newpage 

\section{Processing Notes} 

\begin{itemize} 
\item Table 
ef{chidepinfo} gives the info for $\item Table 
ef{procsum} gives a summary of the processing from MakeCasts. 
\end{itemize} 

% make and insert table 1 

% make and insert table 2 

% continue... 

\subsubsection{Time Offsets} 

\begin{figure}[htbp] 

\includegraphics[scale=1]{FLEAT_timeoffsets_all.png} 
\caption{Time-offsets for all $\chi$pods, found by aligning with CTD data.} 
\label{toffs} 
\end{figure} 

\newpage 

\section{Example Raw Chipod Data for 1 cast} 

\begin{figure}[htbp] 
\includegraphics[scale=0.7]{SNSN2014_001_Fig1_RawChipodTS.png} 
\caption{Raw chipod data from SNSN2014for a CTD cast.} 
\label{snSN2014_1} 
\end{figure} 

\begin{figure}[htbp] 
\includegraphics[scale=0.7]{SNSN2002_001_Fig1_RawChipodTS.png} 
\caption{Raw chipod data from SNSN2002for a CTD cast.} 
\label{snSN2002_1} 
\end{figure} 

\begin{figure}[htbp] 
\includegraphics[scale=0.7]{SNSN1013_001_Fig1_RawChipodTS.png} 
\caption{Raw chipod data from SNSN1013for a CTD cast.} 
\label{snSN1013_1} 
\end{figure} 

\newpage 

\section{Results} 

\clearpage 

\subsubsection{CTD Data } 
\begin{figure}[htbp] 
\includegraphics[scale=0.7]{FLEAT_ctd_t_s.png} 
\caption{Plot of temperature and salinity from CTD downcasts on all casts.} 
\label{} 
\end{figure} 

\clearpage 

\subsubsection{Data from each individual Chipod} 

\begin{figure}[htbp] 
\includegraphics[scale=0.7]{XC_SNSN2014_Vs_lat_AllVars.png} 
\caption{All chipod profiles from sensor SNSN2014. Variables are: N2, dTdz, chi, eps, and KT.} 
\label{} 
\end{figure} 

\begin{figure}[htbp] 
\includegraphics[scale=0.7]{XC_SNSN2002_Vs_lat_AllVars.png} 
\caption{All chipod profiles from sensor SNSN2002. Variables are: N2, dTdz, chi, eps, and KT.} 
\label{} 
\end{figure} 

\begin{figure}[htbp] 
\includegraphics[scale=0.7]{XC_SNSN1013_Vs_lat_AllVars.png} 
\caption{All chipod profiles from sensor SNSN1013. Variables are: N2, dTdz, chi, eps, and KT.} 
\label{} 
\end{figure} 

\clearpage 
\newpage 
\subsubsection{One variable from all Chipods} 

\begin{figure}[htbp] 
\includegraphics[scale=0.7]{FLEAT_chi_AllSNs_Vslat.png} 
\caption{Plot of one variable from all chipods.} 
\label{} 
\end{figure} 

\begin{figure}[htbp] 
\includegraphics[scale=0.7]{FLEAT_KT_AllSNs_Vslat.png} 
\caption{Plot of one variable from all chipods.} 
\label{} 
\end{figure} 

\end{document}  
